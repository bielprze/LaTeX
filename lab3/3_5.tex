\documentclass[a4paper,12pt]{book}      
\usepackage[utf8]{inputenc} 
\usepackage[T1]{fontenc}
\usepackage{times}
\usepackage[polish]{babel}
\usepackage{listings}
\usepackage{color,soul} %ul odpowiada za podkre�lenia
%\usepackage{tipa}
 
\sloppy
 
\lstloadlanguages{TeX}
 
\lstset{
	literate={�}{{\k{a}}}1
           {�}{{\'c}}1
           {�}{{\k{e}}}1
           {�}{{\'o}}1
           {�}{{\'n}}1
           {�}{{\l{}}}1
           {�}{{\'s}}1
           {�}{{\'z}}1
           {�}{{\.z}}1
           {�}{{\k{A}}}1
           {�}{{\'C}}1
           {�}{{\k{E}}}1
           {�}{{\'O}}1
           {�}{{\'N}}1
           {�}{{\L{}}}1
           {�}{{\'S}}1
           {�}{{\'Z}}1
           {�}{{\.Z}}1,
	basicstyle=\small\ttfamily,
	%extendedchars=true
  keywordstyle=\color{red}\ttfamily\bfseries\small,
  %keywordstyle=[2]\color{blue}\ttfamily\underline\small,
  morekeywords={je�eli, to, w, przeciwnym, razie, koniec, warunku},
  %keywords=[2]{numer, karty, kredytowej}
}
\sloppy
\begin{document}

 \setul{0.5ex}{0.1ex}
    \definecolor{Blue}{rgb}{0,0.0,1}
    \setulcolor{Blue}

\begin{lstlisting}[escapechar=\%]
je�eli %\color{blue} \ul{numer karty kredytowej}% jest wa�ny to
	wykonanie transakcji %w% oparciu o %numer karty% i zam�wienie
w przeciwnym razie
    wy�wietlenie wiadomo�ci o niepowodzeniu
koniec warunku
\end{lstlisting}
\end{document}
%https://tex.stackexchange.com/questions/265793/how-can-i-avoid-the-use-of-default-styles-when-no-listings-language-is-set?rq=1

emphstyle={[2]\underbar},
 emph={[2]numer, karty, kredytowej},