\documentclass[a4paper,11pt]{article}
\usepackage[polish]{babel}	%koniczny do polskich znaków
\usepackage[utf8]{inputenc}   % lub utf8
\usepackage[T1]{fontenc}
\usepackage{graphicx}
\usepackage{anysize}
\usepackage{enumerate}
\usepackage{times}
\usepackage{cite}
 
%\marginsize{left}{right}{top}{bottom}
\marginsize{3cm}{3cm}{3cm}{3cm}
\sloppy
 
\begin{document}
\bibliographystyle{abbrv}
Z każdym działającym systemem komputerowym powiązane jest oczekiwanie 
{\em poprawności} jego działania (\cite{Sommerville:2006:SE:1196763}). Istnieje szeroka 
klasa systemów, dla których poprawność powiązana jest nie tylko z 
wynikami ich pracy, ale również z~czasem, w~jakim wyniki te są 
otrzymywane. Systemy takie nazywane są {\em systemami czasu 
rzeczywistego}, a~ponieważ są one rozpatrywane  w~kontekście swojego 
otoczenia, często określane są terminem {\em systemy wbudowane} 
(\cite{Sommerville:2006:SE:1196763}, \cite{Szmuc:etal:MFwIO:15}). 

Ze względu na specyficzne cechy takich systemów, weryfikacja jakości 
tworzonego oprogramowania oparta wyłącznie na jego testach jest 
niewystarczająca. Coraz częściej w~takich sytuacjach, weryfikacja 
poprawności tworzonego systemu lub najbardziej istotnych jego 
modułów prowadzona jest z~zastosowaniem metod formalnych 
\cite{Alur:1990:AMR:90397.90438, Szmuc:etal:MFwIO:15}. 

\bibliography{bibliografia}

\end{document}


%etykiety na początku nie mo że miec spacji, zakończona przecinkiem, po ostatniej pozycji nie dajemy %przecinka