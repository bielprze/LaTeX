\documentclass[a4paper,11pt]{report}
\usepackage[polish]{babel}	%konieczny do polskich znaków
\usepackage[latin2]{inputenc}   % lub utf8
\usepackage[T1]{fontenc}
\usepackage{graphicx}
\usepackage{anysize}
\usepackage{enumerate}
\usepackage{times}
%\usepackage{color}
\usepackage[dvipsnames]{xcolor}
\usepackage{listings}
\lstloadlanguages{C}
\usepackage{titlesec}
\usepackage{txfonts} 
\usepackage{float}
\usepackage{wrapfig} 
 \usepackage{caption}
%\usepackage[framemethod=tikz]{mdframed}

%\marginsize{left}{right}{top}{bottom}
\marginsize{2.5cm}{2.5cm}{2.5cm}{2.5cm}
\sloppy


\definecolor{darkred}{rgb}{0.6,0,0} %rgb 0-1
\definecolor{grey}{rgb}{0.4,0.4,0.4}
\definecolor{orange}{rgb}{1,0.6,0.05}
\definecolor{darkgreen}{rgb}{0.2,0.5,0.05}
\definecolor{darkyellow}{RGB}{218,165,32}%RGB 0-255
%\definecolor{lol){RGB}{0,206,209}



\titleformat{\section}
  {\normalfont\Large}{\thesection}{1em}{}[{\titlerule[0.8pt]}]
 
 \titleformat{\subsection}
  {\normalfont\Large\bfseries}{\thesection}{1em}{}
  
  \newcommand\digitstyle{\color{grey}}
\makeatletter
\newcommand{\ProcessDigit}[1]
{%
  \ifnum\lst@mode=\lst@Pmode\relax%
   {\digitstyle #1}%
  \else
    #1%
  \fi
}
\lstset{literate=
    {0}{{{\ProcessDigit{0}}}}1
    {1}{{{\ProcessDigit{1}}}}1
    {2}{{{\ProcessDigit{2}}}}1
    {3}{{{\ProcessDigit{3}}}}1
    {4}{{{\ProcessDigit{4}}}}1
    {5}{{{\ProcessDigit{5}}}}1
    {6}{{{\ProcessDigit{6}}}}1
    {7}{{{\ProcessDigit{7}}}}1
    {8}{{{\ProcessDigit{8}}}}1
    {9}{{{\ProcessDigit{9}}}}1
    {<=}{{\(\leq\)}}1,
    morestring=[b]",
    morestring=[b]',
    morecomment=[l]//,
}
 
\lstset{language=C,
basicstyle=\ttfamily,
keywordstyle=\color{darkred}\ttfamily\bfseries\large,
keywordstyle=[2]\color{blue}\ttfamily\bfseries\large,
keywordstyle=[3]\color{darkyellow}\ttfamily\bfseries\large,
keywordstyle=[4]\color{darkgreen}\ttfamily\bfseries\large,
stringstyle=\color{darkred}\ttfamily\large,
commentstyle=\color{blue!40!white}\ttfamily\small,
%numberstyle=\tiny\color{grey},
%numbers=left,                    
%numbersep=5pt,
keepspaces=true,
identifierstyle=\color{black}\ttfamily\large,
showstringspaces=false,
aboveskip=0pt,
belowskip=0pt,
extendedchars=true,
%columns=fullflexible,
xleftmargin=0.4cm,frame=TlBr,framesep=5pt,
backgroundcolor=\color{blue!3!white}, 
rulecolor=\color{blue!70!white},
tabsize=5,
keywords=[2]{main},
%keywords=[3]{include},
keywords=[4]{for, return},
%classoffset=1,
%otherkeywords={>,<,include},
 %morekeywords={>,<,include},
 %keywordstyle=\color{darkyellow},
  %classoffset=0,
morekeywords={},
%literate=
 %           *{\{}{{\textcolor{black}{\{}}}{1}
  %          {\}}{{\textcolor{black}{\}}}}{1}
   %         {[}{{\textcolor{black}{[}}}{1}
            % {$#$}{{\textcolor{yellow}{$#$}}}{1},
escapechar=\@,
float=th,
captionpos=t,
}
 
\begin{document}
\chapter*{C/Tablice}
\section*{Wst�p}
\subsection*{Sposoby deklaracji tablic}

\begin{wrapfigure}[1]{o}{0.3\textwidth}
\captionsetup{font=footnotesize,labelfont=footnotesize}
  \centering

 \includegraphics[width=0.3\textwidth]{arr.png}
 \caption*{tablica 10-elementowa}
\end{wrapfigure}

Tablic� deklaruje si� w nast�puj�cy spos�b:
\begin{lstlisting}[linewidth=10cm,float=!h,caption={Deklaracja tablicy},label=type]
 typ nazwa_tablicy[rozmiar];
\end{lstlisting}
gdzie rozmiar oznacza ile zmiennych danego typu mo�emy zmie�ci� w tablicy. Zatem aby np. zadeklarowa� tablic�, mieszcz�c� 20 liczb ca�kowitych mo�emy napisa� tak:
\begin{lstlisting}[float=!h,caption={Przyk�ad deklaracji tablicy},label=exampleOfDeclaration]
 int tablica[20];
\end{lstlisting}
Podobnie jak przy deklaracji zmiennych, tak�e tablicy mo�emy nada� warto�ci pocz�tkowe przy jej deklaracji. Odbywa si� to przez umieszczenie warto�ci kolejnych element�w oddzielonych przecinkami wewn�trz nawias�w klamrowych:
\begin{center}\vspace{-0.4cm}
\begin{lstlisting}[float=!h,caption={Deklaracja z inicjalizacj�},label=declarationAndInicjalization]
 int tablica[3] = {0,1,2};
\end{lstlisting}
\end{center}\vspace{-0.8cm}
Niekoniecznie trzeba podawa� rozmiar tablicy, np.:
\begin{center}\vspace{-0.5cm}
\begin{lstlisting}[float=!h,caption={Deklaracja bez podania rozmiaru tablicy},label=declarationwithoutsize]
 int tablica[] = {1, 2, 3, 4, 5};
\end{lstlisting}
\end{center}\vspace{-0.8cm}
W takim przypadku kompilator sam ustali rozmiar tablicy (w tym przypadku - 5 element�w).
\\[3mm]
\noindent Rozpatrzmy nast�puj�cy kod:\vspace{4cm}%[float=th]
\begin{center}
\begin{lstlisting}[float=!ht,caption={Przyk�ad},label=example]
 @\color{darkyellow}\#include \color{BlueGreen}$<$stdio.h$>$@
 @\color{darkyellow}\#define ROZMIAR 3@
 int main()
 {
   int tab[ROZMIAR] = {3,6,8};
   int i;
   puts ("Druk tablicy tab:");
 
   @\color{darkgreen}{for}@ (i=0; i<ROZMIAR; ++i) {
     printf ("Element numer %d = %d\n", i, tab[i]);
   }
   @\color{darkgreen}{return} 0;
 }
\end{lstlisting}
\end{center}\vspace{-0.8cm}
Wynik:				%[float,floatplacement=H]
\begin{center} \vspace{-0.4cm}
\begin{lstlisting}[float=!ht,caption={Wynik uruchomienia programu},label=output]
 Druk tablicy tab:
 Element numer @0 = 3@
 Element numer @1 = 6@
 Element numer @2 = 8@
\end{lstlisting}
\end{center}
\end{document}


%https://www.sharelatex.com/learn/Code_listing#Using_listings_to_highlight_code
%https://tex.stackexchange.com/questions/117533/is-it-possible-to-use-a-non-monospace-font-in-verbatim-environments
%https://tex.stackexchange.com/questions/27663/using-bold-italic-text-inside-listings
%https://tex.stackexchange.com/questions/119863/problems-with-the-lstlisting-environment-margin-and-white-line
%https://latex.org/forum/viewtopic.php?t=7386

 
 %\surroundwithmdframed[
 % hidealllines=true,
  %backgroundcolor=gray,
  %innerleftmargin=15pt,
  %innertopmargin=0pt,
  %innerbottommargin=0pt]{lstlisting}
  
  %https://en.wikibooks.org/wiki/LaTeX/Source_Code_Listings
  
 % https://tex.stackexchange.com/questions/283170/using-different-colors-for-different-keywords-in-lstlisting
%https://tex.stackexchange.com/questions/32174/listings-package-how-can-i-format-all-numbers
%https://tex.stackexchange.com/questions/34896/coloring-digits-with-the-listings-package
%https://tex.stackexchange.com/questions/23634/how-can-i-change-the-color-of-digits-when-using-the-listings-package
%http://www.tex.ac.uk/FAQ-figurehere.html
%https://tex.stackexchange.com/questions/283170/using-different-colors-for-different-keywords-in-lstlisting