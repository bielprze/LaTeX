\documentclass[a4paper,12pt]{book}
\usepackage[polish]{babel}	%konieczny do polskich znak�w
\usepackage[latin2]{inputenc}   % lub utf8
\usepackage[T1]{fontenc}
\usepackage{graphicx}
\usepackage{anysize}
\usepackage{enumerate}
\usepackage{times}
\usepackage{tipa}
\usepackage{tipx}
\usepackage{extsizes}

%\marginsize{left}{right}{top}{bottom}
\marginsize{3cm}{4.5cm}{3.2cm}{2.5cm}
%\sloppy

\begin{document}
\begin{itemize}
%\textschwa to inaczej@, \textesh to S
\item \textbf{addition} \textipa{[@\textprimstress dIS@n]} -- dodanie, dodatek, dodawanie
\item \textbf{feedback} \textipa{[\textprimstress fi:db\ae k]} -- opinia, rada, odzew, reakcja, odpowied�, sprz�enie \\zwrotne
\item \textbf{grasp} \textipa{[grA:sp]} -- (u)chwyt, chwyta�, pojmowanie, pojmowa�
\item \textbf{origin} \textipa{[\textprimstress OrIdZIn]} -- pocz�tek, �r�d�o, pochodzenie (of person)



\end{itemize}
\end{document}

\textipa{[\!b] [\:r] [\;B]}\quad{\tipasafemode
$ a\:a\quad b\;b\quad c\!c\quad\| $}\quad
\textipa{[\!b] [\:r] [\;B] (back again!)}

http://home.agh.edu.pl/~mszpyrka/lib/exe/fetch.php?media=lectures:latex:tipaman.pdf