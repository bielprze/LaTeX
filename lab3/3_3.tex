\documentclass[a4paper,11pt]{article}
\usepackage[polish]{babel}	%konieczny do polskich znak�w
\usepackage[latin2]{inputenc}   % lub utf8
\usepackage[T1]{fontenc}
\usepackage{graphicx}
\usepackage{anysize}
\usepackage{enumerate}
\usepackage{times}
\usepackage{xcolor}
\usepackage{listings}
\lstloadlanguages{C++}

%\marginsize{left}{right}{top}{bottom}
\marginsize{2.5cm}{2.5cm}{2.5cm}{2.5cm}
\sloppy

\definecolor{darkred}{rgb}{0.9,0,0}
\definecolor{grey}{rgb}{0.4,0.4,0.4}
\definecolor{orange}{rgb}{1,0.6,0.05}
\definecolor{darkgreen}{rgb}{0.2,0.5,0.05}

 
\begin{document}
Silnia \#1:

\lstset{language=C++,
basicstyle=\ttfamily\small,
keywordstyle=\color{darkgreen}\ttfamily\bfseries\small,
stringstyle=\color{red}\ttfamily\small,
commentstyle=\color{grey}\ttfamily\small,
numbers=left,
numberstyle=\color{darkred}\ttfamily\scriptsize,
identifierstyle=\color{blue}\ttfamily\small,
showstringspaces=false,
morekeywords={
}}
%float to wstawka
\begin{lstlisting}[float=th, aboveskip=0pt, belowskip=2pt, extendedchars=true, frame=tRBl, frameround=ftff, backgroundcolor=\color{blue!10!white}]
#include <iostream>
using namespace std;

int main()
{
  int s, n, i;
  cout << "Prosz� poda� liczb� naturaln� n : ";
  cin >> n;
  i = 1;
  s = 1;
  
  while (i < n)
  {
    i++;
    s *= i;
  }
  cout << "Silnia: " << s << endl;
}
\end{lstlisting}
Silnia \#1:
\begin{lstlisting}[float=th, aboveskip=0pt, belowskip=2pt, extendedchars=true, frame=tRBl, frameround=ftff, backgroundcolor=\color{blue!5!white}]

#include <iostream>
using namespace std;

int main()
{
  int s, n, i;  
  cout << "Podaj liczb� naturaln�, n = ";
  cin >> n;
  s = 1;  
  
  for(i = 2; i <= n; i++) s *= i;  
  
  cout << "Silnia n: " << s << endl;
}
\end{lstlisting}
\end{document}
